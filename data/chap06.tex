\chapter{共振光数能同传系统视场角增强方法}
\label{cha:4}

\section{引言}

利用共振光传输系统来实现数能同传,虽然有着高安全性以及可移动自对准的特性,但是目前共振光数能同传系统任然有很多需要发展的地方。
其中在固定发送端的情况下,接收端的可移动范围即系统视场角是制约系统发展的一个关键因素。
本章节首先提出了共振光数能同传系统的具体的系统结构设计,以及系统关键性能仿真的方法。
然后从三个不同的优化角度分别提出了不同的系统视场角增强方法,分别是基于光学像差补偿的视场角增强方法、基于腔内望远镜的视场角增强方法以及基于可旋转发送端的视场角增强方法,并通过理论仿真与实验验证证明了本文提出的视场角增强方法的可靠性。
最后基于现有的视场角增强方法的不足对未来的共振光数能同传系统提出了展望。


\section{基于光学像差补偿的视场角增强方法}

共振光系统主要依赖于分离式谐振腔产生的腔内共振光作为能量和信号的传输载体,分离式谐振腔由收发端的两个回复反射器组成,其结构如图 所示。
其中$p$点是回复反射器的光瞳,位于透镜L的前焦点。
根据ABCD矩阵分析可以知道,在理想情况下,在光瞳位置从任意角度射入回复反射器的光都会沿着相反的方向从光瞳位置反射回来,这就是回复反射器独特的特性之一。
然而由于场区、球差等光学相差的存在,从边缘视场入射的光并不能很好的达到这个效果,这就会导致工作在在边缘视场情况下的共振光系统无法维持稳定,限制了系统的视场角。

本节首先介绍了传统的共振光数能同传系统的结构、数能分离方案、能量接收方案、通信调制以及解调方法。
然后基于提出的仿真模型,搭建了了共振光数能同传系统的通信、传能和视场角的评价模型。
最后基于共振光数能同传系统设计,提出了基于球差优化以及场区优化的视场角增强方法。


\subsection{共振光数能同传系统设计}

谐振束 SLIPT 系统基于SSLR 和二次谐波发生 (SHG) 方案。
利用倍频波束进行通信是抑制回波干扰对通信质量影响的有效策略。
如图 2 所示,功率通过 1064 nm 谐振光束传输,而信息通过 532 nm 倍频光束传输。
该倍频光束由 SHG 晶体的谐振光束产生,并由电光调制器 (EOM) 进行调制。
 1064 nm 谐振光束穿过增益介质经反射镜 M3 反射,从发射器传播到接收器,而 532 nm 倍频光束穿过 EOM 和反射镜 M3,并被反射镜 M4 反射。
 在接收器处,一部分谐振光束被允许穿过镜子 M2 和二向色镜 M5,并随后通过透镜 L4 聚焦在 PV 面板上。
 最后,提取的谐振光束通过光伏发电转化为电能,为电池充电。
 同时,倍频光束穿过反射镜M2,随后被M5与谐振光束分离。
 最后,它通过透镜 L4 集中在光电二极管 (PD) 上,并在此转换为电流信号。
 发射器处的回归反射器由反射镜M1和透镜L1组成,而接收器处的回归反射器由反射镜M2和透镜L2组成。


\subsection{共振光数能同传系统传能通信性能评价模型}

\subsection{基于球差消除的视场角增强方法}

\subsection{基于场区优化的视场角增强方法}


\section{基于腔内望远镜的视场角增强方法}

\section{基于可旋转发送端的视场角增强方法}

\section{小结}